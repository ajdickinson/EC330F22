\documentclass[12pt]{article}

\usepackage{times}

% \usepackage[utf8]{inputenc}
\usepackage{amsmath}
\usepackage{comment}
\usepackage{hyperref}
% \usepackage{subcaption}
% \usepackage{mathrsfs}
\usepackage{amssymb}
% \usepackage{ulem}
% \usepackage{amsthm}
\usepackage{booktabs}
% \usepackage{wrapfig}
\usepackage{tabu}
\usepackage[group-separator={,},group-minimum-digits=4]{siunitx}
\usepackage[mathscr]{euscript}
% \usepackage{tikz}
\allowdisplaybreaks
\usepackage{mathtools}
\usepackage{bm} 
\DeclarePairedDelimiter\abs{\lvert}{\rvert}%
\DeclarePairedDelimiter\norm{\lVert}{\rVert}%
\newcommand{\floor}[1]{\left\lfloor #1 \right\rfloor}
\newcommand{\ceil}[1]{\left\lceil #1 \right\rceil}
\makeatletter
\let\oldabs\abs
\def\abs{\@ifstar{\oldabs}{\oldabs*}}
\let\oldnorm\norm
\def\norm{\@ifstar{\oldnorm}{\oldnorm*}}
\makeatother
\newcommand*{\Value}{\frac{1}{2}x^2}
\usepackage[margin=1in]{geometry}
\setlength{\parskip}{0em}
\allowdisplaybreaks 
\usepackage[margin=1cm]{caption}
\usepackage{xcolor}

\usepackage{soul}
\usepackage{natbib}
\usepackage{import}
\usepackage{xifthen}
\usepackage{pdfpages}
\usepackage{transparent}
\usepackage{threeparttable}
\usepackage{rotating}
\usepackage{graphics}
\usepackage[capposition=top]{floatrow}
\usepackage{float}
\usepackage[bottom]{footmisc}

\newcommand{\incfig}[1]{%
    \def\svgwidth{\columnwidth}
    \import{./figures/}{#1.pdf_tex}
}
\usepackage{setspace}
\hypersetup{
    colorlinks=true,
    citecolor=black,
    linkcolor=black,
    filecolor=black,      
    urlcolor=blue,
    pdfpagemode=FullScreen,
    }



\topmargin 0.0cm
\oddsidemargin 0.2cm
\textwidth 16cm 
\textheight 21cm
\footskip 1.0cm


\title{Heterogenous Responses to Air Quality Alerts} 

\author{Andrew Dickinson}

% Include the date command, but leave its argument blank.
\date{January 31, 2023}

%%%%%%%%%%%%%%%%% END OF PREAMBLE %%%%%%%%%%%%%%%%


\begin{document}

% Double-space the manuscript.
% \baselineskip24pt

% Make the title.
\maketitle

\begin{abstract}
\noindent To reduce exposure to ambient air pollution, policymakers worldwide are increasingly relying on information-based alert systems to inform the public of poor air quality. While a few studies have found these programs to be effective, they are limited by their measure of pollution exposure. I propose to use a rich data source that monitors cellphone movements at fine spatial and temporal resolutions to create a new measure of pollution exposure, updating what has been done with an emphasis on environmental justice via heterogeneous analysis.
\end{abstract}

\newpage 
%%%%%%%%%%%%%%%%%%%%%%%%%%%
%%%%%%%%%%%%%%%%%%%%%%%%%%%
%%%%%%%%%%%%%%%%%%%%%%%%%%%
%%%%%%%%%%%%%%%%%%%%%%%%%%%
%%%%%%%%%%%%%%%%%%%%%%%%%%%
%%%%%%%%%%%%%%%%%%%%%%%%%%%
%%%%%%%%%%%%%%%%%%%%%%%%%%%
%%%%%%%%%%%%%%%%%%%%%%%%%%%
%%%%%%%%%%%%%%%%%%%%%%%%%%%
%%%%%%%%%%%%%%%%%%%%%%%%%%%
%%%%%%%%%%%%%%%%%%%%%%%%%%%
%%%%%%%%%%%%%%%%%%%%%%%%%%%

\section{Conception and definition of the project}

% Ambient air pollution is one of the transcendent problems that affect global health and well-being. It is responsible for more than 4 million deaths each year due to an increased risk of respiratory diseases, heart attacks, and strokes (Landrigan et al., 2017). Furthermore, by 2060 the number of deaths is expected to rise (1). Thus m
Limiting the negative effects of air pollution on human health is a key priority of policymakers for the foreseeable future--particularly in urban areas. One strategy to combat this pollution is to mitigate exposure. Air quality alert programs nudge avoidance behaviors during spells of poor air quality and are increasingly popular worldwide. In the United States, more than 200 cities operate "action day" programs in conjunction with the EPA that create and distribute alerts. Generally, when the local Air Quality Index (AQI) is forecasted to exceed some threshold, an advisory is issued urging individuals to limit outdoor exposure. However, there is variation in the forecast type (i.e., 1-day, 2-day, and 3-day forecasts) and the threshold level across programs. \\

These information-based alert policies are becoming an increasingly popular tool among policymakers as they, if effective, are a relatively cheap way to reduce health costs associated with air pollution. Thus understanding the efficacy of these alert systems is important. Furthermore, in the broader literature that examines the relationship between pollution and health, failing to account for avoidance behavior induced by an alert will bias the estimates of air pollution downward \citep{neidell2004air}. Thus insights into the behavioral response generated by these alerts are consequential. \\

Even though alert programs are increasingly popular, the literature investigating them is thin \citep{zivin2009days, saberian2017alerts}. In general, previous studies find that air quality alerts increase avoidance behaviors. However, the literature is limited in three ways. First, previous studies have concerns about external validity. Limited by data on particular groups, such as attendees to the Los Angeles Zoo and the Griffith Observatory in LA County \citep{zivin2009days} or bicycle commuters in Sydney, Australia \citep{saberian2017alerts}, I have concerns about whether the reductions in ticket sales or bicycle commuters generalize enough to suggest these programs reduce pollution exposure across broader groups in the population. Second, more must be understood about the heterogeneous responses across important dimensions such as income, race, age, and political ideology. Given the expanding environmental justice (EJ) literature documenting environmental inequalities among historically marginalized groups, reevaluating these systems may provide important information to policymakers. Furthermore, studies in adjacent literature evaluating other avoidance behaviors find meaningful differences across groups--particularly by income and race \citep{burke2022exposures, holloway2022unequal}. Third, following scientific advances, the National Ambient Air Quality Standards (NAAQS) have evolved in the last three decades, changing AQI classifications in 1997 and 2005. Prior studies evaluating alert programs in the United States use data from the 1990s, before these changes \citep{zivin2009days}. Additionally, total emissions and air pollutant concentration averages have fallen in the past several decades, and there has been an increase in attention to new categories of pollution like fine particulate matter (EPA). Therefore the behavioral responses to these programs may have changed. \\

Given these concerns, these air quality alert systems are worthy of further research in my research agenda. Do contemporary air quality alert programs still reduce pollution exposure in the United States? Are historically marginalized groups, who already face greater environmental inequalities, equally served by these alert systems? Is the exposure reduction cost-effective? Should the federal government implement a national standard of air quality alerts to reduce the impact of ambient air pollution on the United States healthcare system? To my knowledge, answers to these questions remain unanswered in scientific literature and could provide a fruitful area of interest in my long-term research agenda. I have recently worked with Ed Rubin and Eric Zou on adjacent research interests. \\

My objective for this proposal is to reevaluate the effectiveness of these air quality alert programs in Southern California by linking cellphone-based movement data with Census demographic data at the Census Block Group (CBG) level. These cellphone-movement data monitor `visits' across thousands of Points-of-Interests (POIs) with which I plan to create a measure of avoidance behavior. Southern California is an ideal research setting due to its large number of EPA air quality monitors, consistent 1-day forecasts, and the high number of air quality alerts. Following a similar strategy used by Graff Zivin and Neidell (2009), I aim to identify a causal estimate of these programs on avoidance behavior by leveraging alert system cutoffs in a regression discontinuity (RD) framework--comparing measures of avoidance behaviors between days which are forecasted to slightly exceed the alert system threshold to those that are forecasted to be slightly below. \\

% Since my initial idea to start this project occurred in my second year coursework as a submission for an assigned "grant proposal," I have started collaborating with Ed Rubin and Eric Zou in the same research area as this proposal. Though this proposal contains only my work on this idea, all errors are mine. \\

In the following sections, I briefly describe the current literature in this area, my proposed contributions, and a detailed work plan, including a detailed description of my methodology.

\section{Significance of the project}

To my knowledge, two existing papers link air quality alerts to avoidance behavior. Graff Zivin and Neidell (2009) provide the most compelling evidence in the United States that air quality alert programs lead to increased avoidance behavior. The paper estimates avoidance behavior responses induced by smog alerts\footnote{~A type of air quality alert in California in the 1990s.} by identifying changes across the alert threshold with a regression discontinuity design. The authors use turnstile attendance data between 1989 and 1997 of two popular outdoor venues in Los Angeles County--the Los Angeles Zoo and the Griffith Observatory. They show reductions in attendance to the zoo and observatory by 15 and 8 percent, respectively. 
% However, estimates of consecutive days with issued alerts are not statistically distinguishable from zero, arguing that the intertemporal costs of forgoing an additional day of outdoor time increase on the second day of an alert. 
More recently, Saberian et al. (2017) estimate the effectiveness of air quality alerts among cyclists in Australia between 2008 and 2013. Using administrative data of the cycle path network in Sydney, they find a 30 percent reduction in the number of cyclists on days with an air quality alert. \\

I anticipate for the proposed study to have the following contributions: (i) It will estimate the impact of alert systems upon a more general sample of the population, contributing external validity to the literature; (ii) It will contribute a large set of estimates of heterogeneous groups with the specific aim of understanding the impacts of these programs on historically marginalized groups; (iii) The results will update estimates to the modern alert system used currently in the United States. \\

The first intended contribution of this proposal to the literature is to estimate the effect of air quality alerts on a higher quality proxy for avoidance behavior. Both prior studies have used a limited measure of avoidance behavior that reduces the external validity of their estimates. Attendance tallies to outdoor venues, and measurements of bicycle path traffic only partially represent the general population. Especially for at-risk groups whose exposure to ambient air pollution might be most damaging. For example, individuals that attend a zoo or an observatory may overrepresent higher income populations or families with children. Additionally, bicycle commuters or enthusiasts may overrepresent healthier and more environmentally conscious population groups who may respond fervently to issued alerts. I plan to improve on these measures with a richer data source. \href{https://docs.safegraph.com/docs/weekly-patterns}{Safegraph} 's aggregated cellphone-movement data monitors visits made by millions of cell devices to thousands of Points-of-Interests (POIs). With these data, I can replicate the attendance measures of the LA Zoo and Griffith Observatory of Graff Ziven and Neidell (2009) and expand to thousands of other POIs such as parks, malls, sporting events, and other outdoor venues. Moreover, I aim to filter outdoor venues more accessible to lower income populations or close to minority neighborhoods, constructing a more representative outcome variable. \\

Another contribution in my planned research that may prove helpful is exploring the heterogeneous responses to these alerts across demographic, socioeconomic, and political groups. Safegraph uses an internal model to predict home CBG for each cellphone. Thus the dataset contains the number of visits to each POI by home CBG, allowing for observations to be merged with Census demographic information. Thus one can observe the number of individuals that visit the Zoo from each CBG at a fine temporal resolution. I plan to use these data to estimate the treatment effect of air quality alerts across CBGs across dimensions including income, race, and political ideology. The final contribution is updating estimates for the modern alert systems in the United States. To my knowledge, the estimates found in Graff Zivin and Neidell (2009) from the 1990s are the most recent evidence in the literature of the impact of alerts on avoidance behavior.


\section{Plan of work and methodology}

As evidenced above, I have made progress in a literature review in the immediate area of this research question. Up to this point, I have successfully downloaded air quality data from the EPA Air Quality System (AQS) throughout the United States since 2012, including information on forecasted AQI, forecast dates, measured AQI, and whether an air quality alert was issued. In addition, I have downloaded demographic and socioeconomic data at the CBG level from the American Community Survey (ACS) 5-year estimates from 2019. Thanks to Ed Rubin, I have access to SafeGraph data from 2018 onwards. \\

\noindent My current plan to identify a causal effect is specified below:

\begin{equation*}
\text{Avoidance Behavior}_{it} ~=~ \alpha_0 \text{Alert}_{t} ~+~ \beta_{0} \text{AQI}^{f}_{t} ~+~ X_{it} ~+~ \varepsilon_{it}
\end{equation*}

\noindent where $\text{Avoidance Behavior}_{it}$ is some measure of avoidance behavior from the data for individuals from CBG $i$ in time $t$. To start this analysis, I plan to count the visits to major outdoor POIs and estimate them separately. $\text{Alert}_{jt}$ is a dummy variable indicating whether an air quality alert is issued in time $t$. $\text{AQI}^{f}_{t}$ is the AQI forecast which invokes the RD design at some threshold\footnote{~when AQI changes from "Somewhat Unhealthy" to "Unhealthy"} starting with a linear specification. $X_t$ is a set of year-by-month fixed effects to account for seasonal trends, day-of-week fixed effects to account for within-week variation, and potential confounding variables such as holiday dummies and meteorological variables. $\varepsilon_{it}$ is an error term.  \\

Throughout the Summer, I plan to:
(i) Perform a more exhaustive literature review in the broader avoidance behavior literature, 
(ii)Download all cellphone-movement data and merge it with demographic and socioeconomic information from the ACS and action day alerts from the AQS,
(iii) Test and create a representative proxy for avoidance behavior using the cellphone-movement data that is representative of Southern California,
(iv) Estimate the reduction in ambient air pollution exposure with a regression discontinuity identification strategy,
(v) Perform heterogeneity analysis across group characteristics.
I am confident that downloading and merging the required data can be done relatively quickly. Most of my time over the Summer will be spent building a good proxy for avoidance behaviors and estimating the relationship between it and air quality alerts and writing. I hypothesize that the results will be consistent with prior literature--air quality alerts lead to a reduction in ambient air pollution exposure. However, I think that wealthier and whiter CBGs will have a greater propensity to avoid air pollution than poorer CBGs with a higher composition of minority groups. Suggesting further expansion of these programs along with a greater emphasis on outreach among marginalized groups.

\section{Obligations}

Conditional on receiving this award, I plan to focus exclusively on this project and seek no other summer employment. I plan to meet the residency requirements as well. \\


\clearpage
\singlespace
% \bibliography{DickinsonWaddell_DaylighSaving_REFERENCES.bib}
% \bibliographystyle{apalike}



\bibliography{kleisorge23.bib}
%\bibliographystyle{apalike}
\bibliographystyle{chicago}





\end{document}


